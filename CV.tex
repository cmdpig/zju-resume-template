\documentclass[11pt]{article}


\setlength{\parindent}{0pt}
\usepackage{xltxtra}
\usepackage{hyperref}
\hypersetup{hidelinks}
\usepackage{url}
\urlstyle{tt}
\usepackage{xcolor}
\definecolor{CVBlue}{RGB}{23,110,191}
\usepackage{calc}
\usepackage{graphicx}
\usepackage{tikz}
\usetikzlibrary{calc}
\usepackage{fontspec}
\usepackage{xeCJK}
\usepackage{enumitem}
\CJKsetecglue{} %% 取消中文与数字之间的间隙


%% 主文档字体设置
\setmainfont[
    Path = fonts/Main/,
    Extension = .otf,
    BoldFont = texgyretermes-bold.otf, % 加粗字体
]{texgyretermes-regular.otf} % 正文字体

% 中文字体设置
\setCJKmainfont[
    Path = fonts/hansans/,
    Extension = .ttf,
    BoldFont = NotoSansSC-Bold.ttf, % 加粗字体
]{NotoSansSC-Regular.otf} % 正文字体


\usepackage{relsize}
\usepackage{xspace}

% 使用 fontawesome(部分图标)
\usepackage{fontawesome} 

% A4纸,上下左右边距
\usepackage[
    a4paper,
    left=1.2cm,
    right=1.2cm,
    top=1.5cm,
    bottom=1cm,
    nohead
]{geometry}

\renewcommand{\baselinestretch}{1.5} % 行间距设为1.5

\usepackage{titlesec}
\usepackage{enumitem}
\setlist{noitemsep} % 取消列表项间的额外间距
%\setlist{nosep} % 取消所有垂直间距
\setlist[itemize]{topsep=0.25em, leftmargin=*}
\setlist[enumerate]{topsep=0.25em, leftmargin=*}

% --- 用于控制【不同项目之间】的垂直距离 ---
\newlength{\interProjectSpacing}
\setlength{\interProjectSpacing}{0.9em} % <--- 在此调整项目之间的距离
\newcommand{\projectsep}{\vspace{\interProjectSpacing}}

% --- 用于控制【项目标题】与下方【项目描述】的距离 ---
\newlength{\intraProjectTitleSep}
\setlength{\intraProjectTitleSep}{0.4em} % <--- 在此调整标题和描述的距离
\newcommand{\titlebreak}{\\[\intraProjectTitleSep]}

% --- 用于控制【项目描述】与下方【要点列表】的距离 ---
\newlength{\intraProjectListTopSep}
\setlength{\intraProjectListTopSep}{0.2em} % <--- 在此调整描述和列表的距离

% =======================================================================


\titleformat{\section}         % 定制 \section 命令 
{\large\bfseries\raggedright} % 将 section 标题设置为大号、粗体且左对齐
{}{0em}                      % 可用于为所有 section 添加前缀(如“章节...”)
{}                           % 可用于在标题前插入代码
[{\color{CVBlue}\titlerule}]  % 在标题后插入一条横线
\titlespacing*{\section}{0cm}{*1.6}{*1.2}



\begin{document}
\pagenumbering{gobble}

%%%% 利用tikz来定位照片

%%%% 利用tikz来定位学校Logo,这里只在第一页显示
\begin{tikzpicture}[remember picture, overlay] 
    \node[anchor = north west] at ($(current page.north west)+(0.5cm,+1.0cm)$) {\includegraphics[height=6cm]{zju.png}};
\end{tikzpicture}%
\centerline{\LARGE\bfseries{袁怡帆}} 

\centerline{\normalsize{\faPhone\ 178-6784-7107 \quad \faEnvelopeO\ \href{mailto:3250104281@zju.edu.cn}{3250104281@zju.edu.cn}}} 

\centerline{\normalsize{\faGithubSquare\ \href{https://github.com/cmdpig}{https://github.com/cmdpig} }} 
    
\section{\makebox[\widthof{\faGraduationCap}][c]{\color{CVBlue}\faGraduationCap}\ 教育背景}    
\textbf{浙江大学} \hfill 2025.9 -- 至今\\[0.5em] % 标题和正文间加一点距离
建筑学\quad 大一 

\section{\makebox[\widthof{\faUsers}][c]{\color{CVBlue}\faUsers}\ 个人经历}

% --- 第一个项目 ---
% 将标题行末尾的 \\ 替换为 \titlebreak 命令

% --- 第二个项目 ---
\textbf{UCB CS 161} \hfill 2026.02 -- 至今 \titlebreak
部分参与了加州大学伯克利分校的CS 161课程,深入学习了计算机安全的核心概念和实践技能。通过完成课程项目,掌握了漏洞分析、攻击技术和防御机制等关键领域的知识。
\begin{itemize}[nosep, topsep=\intraProjectListTopSep]
    \item \textbf{课程学习}:截至目前,学习到lecture 8,涵盖了缓冲区溢出、格式化字符串漏洞、密码学等多个重要主题。通过理论学习和实践操作,提升了对计算机安全的理解和应用能力。
    \item \textbf{项目实践}:个人独立完成proj1,全过程记录。相关代码开源在github上
\end{itemize}



\section{\makebox[\widthof{\faGraduationCap}][c]{\color{CVBlue}\faList}\ 获奖情况}
\begin{itemize}
    \item NOIP2023二等奖 \hfill 2023.12
\end{itemize}
    
\section{\makebox[\widthof{\faInfo}][c]{\color{CVBlue}\faInfo}\ 其他}
\begin{itemize}[parsep=0.5ex]
    \item \textbf{技术博客:} \href{https://www.cnblogs.com/cmd-pig}{https://www.cnblogs.com/cmd-pig}
    \item \textbf{GitHub:} \href{https://github.com/cmdpig}{https://github.com/cmdpig} 
\end{itemize}
\end{document}